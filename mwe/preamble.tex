\usepackage{amsmath} % Nice maths symbols.
\usepackage{amssymb} % Nice variable symbols.
\usepackage[english]{babel}
\usepackage{bbm} % Gives Blackboard fonts.
\usepackage[math]{blindtext}
\usepackage{enumitem} % Nice listing options in itemize and enumerate.
\usepackage{fancyhdr}
\usepackage{float} % Nice figure placement.
\usepackage[T1]{fontenc} % Nice range of text characters and accents.
\usepackage{graphicx} \graphicspath{{../.././figures/}{../.././figures/logos/}}
\usepackage[geometry]{ifsym} % Useful symbols.
\usepackage{ifthen}
\usepackage{letltxmacro} % For defining a nice SQRT symbol.
\usepackage{lipsum}
\usepackage{listings} % The listings package for code.
\usepackage{mathtools} % Gives the colon equals symbol.
\usepackage[framed,numbered,autolinebreaks,useliterate]{mcode} % Inports Listings package ideal for MATLAB.
\usepackage{multirow} % Nice table cells spanning many rows.
\usepackage{multicol} % If I want to use multiple columns.
\usepackage[numbers, sort&compress]{natbib} % Nice references.
\usepackage{sansmath} % Gives a changing math font. %% Must be before newpxtext/newpxmath ! %%
\usepackage{newpxtext} % Gives Palatino and Helvetica fonts.
\usepackage{newpxmath} % Gives Palatino and Helvetica fonts in maths. 
\usepackage{bm} % Bold math symbols. %% Needs to be loaded after newpxtext/math. %%
\usepackage[super]{nth} % Gives nice ordinal superscripts, eg 1st, 2nd, etc.
\usepackage{parskip} % Gives nicer indenting.
\usepackage{physics} % Nice partial derivatives and BRAKET notation.
\usepackage{ragged2e}
\usepackage{romannum} % Nice typing for roman numerals.
\usepackage{setspace} % Ideal for increasing line spacing. E.g.  \doublespacing
\usepackage{sidecap} \sidecaptionvpos{figure}{t}
\usepackage{siunitx} % Nice formating of units.
\usepackage{subcaption} % Side by side figures.
\usepackage{tikz} % Nice diagrams.
\usepackage{titling} % Access title variables. 
\usepackage{titlesec} % Nice section title colouring options.
\usepackage{todonotes}
\usepackage{wrapfig}
\usepackage{xcolor} % This is useful for making greyed table cells, nice for headers. Known preamble placement issues.
\usepackage{xparse} 
\usepackage{xspace} % Gives nice spacing for commands.
\usepackage[colorlinks=true,linkcolor=black,urlcolor=black,citecolor=black,anchorcolor=black]{hyperref} % Colour links.
%%%% Should be loaded after hyperref. %%%%
\usepackage{cleveref} % Gives smart referencing. %% After Hyperref
\usepackage[margin=10pt,font=small, textfont=bf,labelfont=bf,labelsep=endash]{caption} % Caption figures and tables nicely. %% After cleveref.
\newlength{\innermarginonesided}
\setlength\innermarginonesided{40mm}
\newlength{\outermarginonesided}
\setlength\outermarginonesided{15mm}
\newlength{\margindifference}
\setlength\margindifference{\innermarginonesided-\outermarginonesided}
\usepackage[top=20mm, bottom=20mm, left=\innermarginonesided, right=\outermarginonesided]{geometry}
\usepackage{cleveref} % Gives smart referencing. %% After Hyperref

% Making an index. 
\makeindex

% Present the references in the order they are used.
\bibliographystyle{unsrtnat}
% Reduce spacing between references. 
\setlength{\bibsep}{0pt plus 0.3ex}

% Listing -> Code in environment labels.
\renewcommand{\lstlistingname}{Code}
\crefname{listing}{code}{codes}
\Crefname{listing}{Code}{Codes}
\lstset{
    numbers=left, 
    basicstyle=\ttfamily\footnotesize,
    frame=single, % adds a frame around the code
    xleftmargin=20.4pt,
    xrightmargin=3.4pt,
    %	numbersep=3mm,
}
\newfloat{lstfloat}{htbp}{lop} % environment for placing lisings in to make them float. 

\renewcommand{\bibname}{References}

% Gives the nice SQRT symbol.
\makeatletter
\let\oldr@@t\r@@t
\def\r@@t#1#2{%
    \setbox0=\hbox{$\oldr@@t#1{#2\,}$}\dimen0=\ht0
    \advance\dimen0-0.2\ht0
    \setbox2=\hbox{\vrule height\ht0 depth -\dimen0}%
    {\box0\lower0.4pt\box2}}
\LetLtxMacro{\oldsqrt}{\sqrt}
\renewcommand*{\sqrt}[2][\ ]{\oldsqrt[#1]{#2}}
\makeatother

% Removes most hyphenation.
\tolerance=1
\emergencystretch=\maxdimen
\hyphenpenalty=10000
\hbadness=10000

\setlength{\parindent}{0em}
\setlength{\parskip}{1.5em}

% To change the spacing in lists:
%\setlist{noitemsep} % or \setlist{noitemsep} to leave space around whole list
% or
\setenumerate{itemsep=-0.2em,topsep=0.5em} % Seems to look nice.

% Custom column widths using C{2cm}, L, R, etc.
\newcolumntype{L}[1]{>{\raggedright\let\newline\\\arraybackslash\hspace{0pt}}m{#1}}
\newcolumntype{C}[1]{>{\centering\let\newline\\\arraybackslash\hspace{0pt}}m{#1}}
\newcolumntype{R}[1]{>{\raggedleft\let\newline\\\arraybackslash\hspace{0pt}}m{#1}}

% Nice spacing in the first row of a table
\newcommand{\firstrowspacing}{\rule{0pt}{2.6ex}}
% For a more open look in tables.
\setlength\extrarowheight{3pt}

% Gives a nice quote environment.
\NewDocumentEnvironment{myquote}{O{}}{%
    \begin{center}
        \begin{minipage}{0.85\linewidth}
            \vspace{1ex}
            \centering \itshape \justifying}
        {%
			\ifx#1
		        % If empty do nothing
		    \else
			    \begin{flushright}%The author/source.
				    \normalfont --#1
			    \end{flushright}
		    \fi
            \vspace{1ex}
        \end{minipage}
    \end{center}
}


% Ensures the SI command detects the local environment (e.g. in captions).
\sisetup{detect-weight=true, detect-family=true}


\newboolean{myfaceset} 
\setboolean{myfaceset}{false}   

\newcommand{\myfacepath}{}
\makeatletter
\newcommand{\myface}[1]{%
    \renewcommand{\myfacepath}{#1}%
    \setboolean{myfaceset}{true}%
}
\makeatother



\fancyhead[LO]{%
    \noindent
    %\hspace*{-\innermarginonesided}
    \begin{wrapfigure}{l}{\innermarginonesided}
        \vspace*{-0.02\paperheight}
        \hspace*{-2.2\outermarginonesided}
        \includegraphics[width=\linewidth]{side_banner}\\
        \hspace*{-2.2\outermarginonesided}
        \includegraphics[width=0.91\linewidth]{epsrc_logo}
        %\end{minipage}
    \end{wrapfigure}
}
\renewcommand*{\headrulewidth}{0pt}%
\cfoot{}


\definecolor{oxfordblue}{RGB}{0, 33, 71}
\newlength{\headerlogoheight}
\setlength{\headerlogoheight}{0.1\paperheight}
\renewcommand{\maketitle}{%
    \pagestyle{fancy}
    \vspace*{-2cm}
        \begin{minipage}[t][\headerlogoheight]{\linewidth}
        \begin{flushright}%
        \includegraphics[height=\headerlogoheight]{infomm_logo_blue}%
    \hspace{0.03\paperwidth}%
    \includegraphics[height=\headerlogoheight]{oxford_logo_blue}%
    \end{flushright}%
    \end{minipage}
    \par 
    \begin{minipage}[t]{\linewidth}
    \centering \Huge \textsc{\textbf{\color{oxfordblue}\thetitle}}
    \end{minipage}
    \par%
    \vspace{0.2\headerlogoheight}
    \begin{wrapfigure}{l}{0.3\linewidth}%
        \vspace{-1.4\baselineskip}
        \begin{minipage}[t]{\linewidth}%[t][0mm][t]{\linewidth}
            \noindent
            \centering
            \ifthenelse{\boolean{myfaceset}}{%
             \includegraphics[width=\linewidth]{\myfacepath}%
             }{%
             \begin{minipage}[t][2\headerlogoheight]{\linewidth}
             \missingfigure[figwidth=\linewidth]{}
             \end{minipage}
             }
             \begin{minipage}[t]{\linewidth}
             \centering
             \large
             \ifthenelse{\boolean{myfaceset}}{}{Specifcy where to find your picture using \texttt{\textbackslash{}myface\{\}}\\}
             \textsc{\textbf{\color{oxfordblue}\theauthor}}
             \end{minipage}   
        \end{minipage}%
        \vspace{-1.4\baselineskip}
    \end{wrapfigure}%
    \par%
}


\titleformat{\section}
{\color{oxfordblue}\normalfont\fontsize{16}{20}\selectfont\bfseries}
{}{0em}{}
\titlespacing\section{0pt}{8pt plus 4pt minus 2pt}{8pt plus 5pt minus 5pt}
